\documentclass{article}
\usepackage{amsmath, amssymb, amsthm}
\usepackage{mathtools}

\newtheorem{theorem}{Theorem}
\newtheorem{lemma}{Lemma}

\begin{document}

\title{Ordering a Subset of $\mathbf F_p^\times$ with Distinct Partial Sums}
\author{}
\date{}
\maketitle

\begin{abstract}
Let $p$ be a prime and let $A\subseteq \mathbf F_p^\times$ be a finite nonempty set. We prove that
there is an ordering of the elements of $A$ whose running sums are pairwise distinct. The argument
packages the obstruction to distinct partial sums into a multivariate polynomial and invokes the
combinatorial Nullstellensatz.
\end{abstract}

Throughout the paper we identify $\mathbf F_p$ with the residue class ring $\mathbf Z/p\mathbf Z$.
For a $t$-tuple $\mathbf x=(x_1,\dots,x_t)$ write $S_j(\mathbf x)=\sum_{k=1}^j x_k$ with the
convention $S_0(\mathbf x)=0$. The tuple has \emph{distinct partial sums} if the values
$S_1(\mathbf x), \dots, S_t(\mathbf x)$ are pairwise distinct.

\section{Statement}
\begin{theorem}
Let $p$ be prime and let $A \subseteq \mathbf F_p^\times$ be finite. Then there exists an ordering
$A = \{a_1,\dots,a_t\}$ of its elements whose partial sums $S_j = \sum_{k=1}^j a_k$ are pairwise
distinct.
\end{theorem}

\section{Encoding the obstruction}
Fix $t = |A| \ge 1$ and consider indeterminates $x_1,\dots,x_t$. For $0 \le i < j \le t$ denote by
\[
  B_{i,j}(x_1,\dots,x_t) = \sum_{k=i+1}^{j} x_k
\]
the sum of the block of $x$'s between positions $i+1$ and $j$. Equal partial sums $S_j(\mathbf x) =
S_i(\mathbf x)$ (with $j>i$) are equivalent to the vanishing of $B_{i,j}(\mathbf x)$.

We separate the factors that merely enforce the nonvanishing of the individual coordinates. Since
our variables will eventually range over $A \subseteq \mathbf F_p^\times$, each coordinate is forced
to be nonzero. Consequently it suffices to record only the factors with $j-i \ge 2$. Define the
``block polynomial''
\begin{equation}
  G(x_1,\dots,x_t) = \prod_{0 \le i < j \le t \atop j-i\ge 2} B_{i,j}(x_1,\dots,x_t)
  \label{eq:block-poly}
\end{equation}
in $\mathbf Z[x_1,\dots,x_t]$. A tuple $\mathbf x \in (\mathbf F_p^\times)^t$ has distinct partial
sums if and only if $G(\mathbf x) \ne 0$.

To also guarantee that the entries are distinct we multiply by the Vandermonde determinant
\[
  \Delta(x_1,\dots,x_t) = \prod_{1 \le i < j \le t} (x_j - x_i).
\]
The product
\begin{equation}
  P(x_1,\dots,x_t) = \Delta(x_1,\dots,x_t) \cdot G(x_1,\dots,x_t)
  \label{eq:poly-def}
\end{equation}
belongs to $\mathbf Z[x_1,\dots,x_t]$. By construction we have $P(\mathbf x) = 0$ whenever two
coordinates of $\mathbf x$ coincide or whenever two partial sums coincide.

\section{A distinguished monomial}
We now exhibit a monomial whose coefficient in $P$ is $\pm 1$. Expanding the Vandermonde determinant
as the determinant of the matrix $[x_i^{j-1}]_{1\le i,j\le t}$ shows that the coefficient of
$x_1^{0} x_2^{1} \cdots x_t^{t-1}$ equals $1$. In the block polynomial $G$ we select the $x_{i+1}$
term from every factor $B_{i,j}$ (this is legitimate because $j-i \ge 2$). The index $i+1$ ranges
from $1$ to $t-1$ and the factor $x_{i+1}$ is chosen once for each $j$ with $i+2 \le j \le t$. Hence
$x_m$ occurs exactly $t-m$ times for $1 \le m \le t$, and the resulting monomial is
$x_1^{t-1} x_2^{t-2} \cdots x_t^{0}$. Therefore $P$ contains the monomial
\begin{equation}
  M(x_1,\dots,x_t) = x_1^{t-1} x_2^{t-1} \cdots x_t^{t-1}
  \label{eq:leading}
\end{equation}
with coefficient $\pm 1$. In particular that coefficient remains nonzero modulo $p$.

\section{Applying the combinatorial Nullstellensatz}
Let each variable $x_i$ range over the finite set $A$. Since $|A| = t$, the inequality
$|A| > t-1$ holds for every index. The monomial identified in~\eqref{eq:leading} therefore satisfies
all hypotheses of the combinatorial Nullstellensatz. Consequently there exist elements
$(a_1,\dots,a_t) \in A^t$ such that $P(a_1,\dots,a_t) \ne 0$.

Because of the Vandermonde factor the coordinates $a_i$ must be pairwise distinct, hence they form a
permutation of the set $A$. The nonvanishing of the block polynomial $G$ implies that none of the
block sums $a_{i+1} + \dots + a_j$ with $j-i \ge 2$ vanishes. Since the elements of $A$ are
nonzero, blocks of length one also have nonzero sum. Thus no two partial sums coincide, and the
ordering $(a_1,\dots,a_t)$ satisfies the desired conclusion.

\section{Conclusion}
Starting from a finite subset of $\mathbf F_p^\times$ we have produced an ordering whose partial
sums are pairwise distinct. The proof relies on a single application of the combinatorial
Nullstellensatz to the polynomial defined in~\eqref{eq:poly-def}.
\qed

\begin{thebibliography}{9}
\bibitem{AlonCNS} N.~Alon, \emph{Combinatorial Nullstellensatz}, Combinatorics, Probability and
Computing\,8 (1999), no.~1--2, 7--29.
\end{thebibliography}

\end{document}
